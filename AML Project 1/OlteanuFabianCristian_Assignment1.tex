\documentclass{article}
\usepackage[a4paper, portrait, margin=1.1811in]{geometry}
\usepackage[english]{babel}
\usepackage[utf8]{inputenc}
\usepackage[T1]{fontenc}
\usepackage{helvet}
\usepackage{etoolbox}
\usepackage{graphicx}
\usepackage{titlesec}
\usepackage{amsfonts}
\usepackage{caption}
\usepackage{booktabs}
\usepackage{xcolor} 
\usepackage[colorlinks, citecolor=cyan]{hyperref}
\usepackage{caption}
\captionsetup[figure]{name=Figure}
\graphicspath{ {./images/} }
\usepackage{scrextend}
\usepackage{amsmath}
\usepackage{fancyhdr}
\usepackage{graphicx}
\newcounter{lemma}
\newtheorem{lemma}{Lemma}
\newcounter{theorem}
\newtheorem{theorem}{Theorem}

\fancypagestyle{plain}{
	\fancyhf{}
	\renewcommand{\headrulewidth}{0pt}
	\renewcommand{\familydefault}{\sfdefault}
}

%\pagestyle{plain}
\makeatletter
\patchcmd{\@maketitle}{\LARGE \@title}{\fontsize{16}{19.2}\selectfont\@title}{}{}
\makeatother

\usepackage{authblk}
\renewcommand\Authfont{\fontsize{10}{10.8}\selectfont}
\renewcommand\Affilfont{\fontsize{10}{10.8}\selectfont}
\renewcommand*{\Authsep}{, }
\renewcommand*{\Authand}{, }
\renewcommand*{\Authands}{, }
\setlength{\affilsep}{2em}  
\newsavebox\affbox
\author{\textbf{Olteanu Fabian Cristian}}
\affil{FMI, AI Master, Year 1
}

\titlespacing\section{0pt}{12pt plus 4pt minus 2pt}{0pt plus 2pt minus 2pt}
\titlespacing\subsection{12pt}{12pt plus 4pt minus 2pt}{0pt plus 2pt minus 2pt}
\titlespacing\subsubsection{12pt}{12pt plus 4pt minus 2pt}{0pt plus 2pt minus 2pt}


\titleformat{\section}{\normalfont\fontsize{10}{15}\bfseries}{\thesection.}{1em}{}
\titleformat{\subsection}{\normalfont\fontsize{10}{15}\bfseries}{\thesubsection.}{1em}{}
\titleformat{\subsubsection}{\normalfont\fontsize{10}{15}\bfseries}{\thesubsubsection.}{1em}{}

\titleformat{\author}{\normalfont\fontsize{10}{15}\bfseries}{\thesection}{1em}{}

\title{\textbf{\huge Advanced Machine Learning Assignment 1}}
\date{}    

\begin{document}

\pagestyle{headings}	
\newpage
\setcounter{page}{1}
\renewcommand{\thepage}{\arabic{page}}


	
\captionsetup[figure]{labelfont={bf},labelformat={default},labelsep=period,name={Figure }}	\captionsetup[table]{labelfont={bf},labelformat={default},labelsep=period,name={Table }}
\setlength{\parskip}{0.5em}
	
\maketitle
	
\noindent\rule{15cm}{0.4pt}
\renewcommand{\thesubsection}{\thesection.\alph{subsection}}

\section{Exercise 1}
\subsection{}
Let $\mathcal{H}$ be our hypothesis class, where:
$
\mathcal{H}=\{h_{\omega_0, \omega_1,\omega_2,...,\omega_{2022}}:\mathbb{R}^{2022} \rightarrow \{0,1\}|h_{\omega_0, \omega_1,\omega_2,...,\omega_{2022}}(x) = \mathbf{1}_{\omega_1x_1 + ... + \omega_nx_n \leq \omega_0}(x), \omega_0,\omega_1,\omega_2,...,\omega_{2022}\in \{1,2\}\}.
$

In other words, $\mathcal{H}$ is a finite class (because components of $\omega$ can only be either 1 or 2) composed of halfspaces in $\mathbb{R}^{2022}$. Furthermore, if we denote $\omega=(\omega_1,...,\omega_{2022})\in \mathbb{R}^{2022}$, the following expressions are equivalent: 
$$
h_\omega(x)=\mathbf{1}_{\omega_1x_1 + ... + \omega_nx_n \leq \omega_0}=\mathbf{1}_{\omega \cdot x \leq \omega_0}=sgn(\omega \cdot x - \omega_0),
$$ where $sgn$ is the signum function.

In order to justify the fact that $VCdim(\mathcal{H})=2023$, we first have to prove that $VCdim(\mathcal{H})\geq2023$ and afterwards $VCdim(\mathcal{H})<2024$. We can consider the standard basis of $\mathbb{R}^{2022}$ plus the origin ($e_0=\mathbf{0}_{2022}$) as a set $B=\{e_0,...,e_{2022}\}$ and prove that it is shattered by $\mathcal{H}$. Given a certain labelling $l_0,...,l_{2022}$ to these points, we set the following relations:
\begin{align*}
\omega_0 &= -l_0, \\
\omega_i &= \omega_0 + l_i, i=\overline{1,2022},
\end{align*}
so $\omega \cdot e_0 - \omega_0=l_0$ and for all $i=\overline{1,2022}$, $\omega \cdot e_i - \omega_0 = l_i$ (for example, $\omega \cdot e_1 - \omega_0 = \omega_1 - \omega_0 = \omega_0 + l_1 - \omega_0 = l_1$. This proves that $B$ is shattered by $\mathcal{H}$ and, since $|B|=2023$, $VCdim(\mathcal{H}) \geq 2023$. Additionally, this statement holds for any $\omega_0,...,\omega_{2022} \in \mathbb{R}$.

The proof of $VCdim(\mathcal{H}) < 2024$ can be achieved by utilizing Radon's Lemma, which states that for a set $S$ from $\mathbb{R}^d$, $|S|=d+2$, there are two subsets of $S$ with the property that their convex hulls intersect. Starting from this theorem, we can construct a set $S={y_1,...,y_{2024}}$ and assign to each element the label 1. Now, if we were to split $S$ according to Radon's Lemma, we would arrive at the conclusion that one point always lies in the convex hulls of both subsets from $S$, so $VCdim(\mathcal{H}) < 2024$ \cite{vcdim2023}.

Thus, it is proven that $VCdim(\mathcal{H})=2023$ for any $\omega\in\mathbb{R}^{2023}$, so the original statement is also valid.
\subsection{}
Let $\mathcal{H}$ be our hypothesis class, where:
$
\mathcal{H}=\{h_{\omega_0, \omega_1,\omega_2,...,\omega_{2022}}:\mathbb{R}^{2022}\in\{0,1\}|h_{\omega_0, \omega_1,\omega_2,...,\omega_{2022}}(x) = \mathbf{1}_{\omega_1x_1 + ... + \omega_nx_n \leq \omega_0}(x), \omega_0, \omega_1,\omega_2,...,\omega_{2022}\in \mathbb{R}\}.
$
$\mathcal{H}$ is infinite because $\omega_0,...,\omega_{2022}\in\mathbb{R}$. $VCdim(\mathcal{H})$ is also 2023 as described in subsection $a$.
\subsection{}
Let $\mathcal{H}=\{ h_\theta:[-1,1]\rightarrow\{0,1\}|h_\theta(x)=\mathbf{1}_{sin(\theta x) \geq 0}(x), \theta \in \mathbb{R}\}$. If we were to consider some set $X=\{x_1,...,x_n\}\subset[-1,1]$, $\mathcal{H}$ can shatter $X$ because $sin(\theta x)$ can oscillate at any frequency to accommodate labeling $X$. Hence, $VCdim(\mathcal{H})=\infty$. 

\section{}
We have $\mathcal{H}=\{ h_a:\mathbb{R}^3\rightarrow \{0,1\}|h_a(x)=\mathbf{1}_{[||x||_2\leq a]}(x), x=(x_1,x_2,x_3)\in\mathbb{R}^3, ||x||_2 = \sqrt{x_1^2+x_2^2+x_3^2} \}$.
\subsection{}
Let's consider the realizability assumption: there exists some $h_a^*\in\mathcal{H}$ such that $L(h_a^*)=0$ ($a^*\in\mathbb{R}$), where $h_a^*(x)=\mathbf{1}_{[||x||_2\leq a^*]}$.

\subsection{}
We know that $VCdim(\mathcal{H})$ is at least greater or equal than 1, since the hypothesis class contains non-constant classifiers, so it is able to shatter any set $S_0=\{x_0\in\mathbb{R}^3\}$, where $|S_0|=1$, regardless of $a$.

Let's take $S=\{x_0, x_1\}\subset\mathbb{R}^3$, $|S|=2$ and set $a=1$.
\newpage

%\bibliography{template} %-->reference list is on the template.bib file
\begin{thebibliography}{1.7} 
	\bibitem[1]{vcdim2023} \color{cyan}Stefan Hausler, VC Dimension, Tutorial for the Course Computational Intelligence, \url{https://www2.spsc.tugraz.at/www-archive/downloads/vc_examples.pdf} \color{black}
	
\end{thebibliography}

\end{document}